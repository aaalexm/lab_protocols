\documentclass{article}
\usepackage[utf8]{inputenc}
\usepackage{enumerate}
% Al subrayar con el comando \hl{} teniendo estos paquetes, se pone en amarillo. Si subrayamos con el comando \underline{} subraya normal.
\usepackage{xcolor}
\usepackage{soul}

% Para seleccionar los márgenes.
\usepackage[a4paper]{geometry}
\geometry{top=2cm, bottom=2.5cm, left=2.5cm, right=2.5cm}

\author{Alejandro Matía}
\date{February 2020}

\begin{document}

% Para poner el título al tamaño deseado.
\font\myfont=cmr12 at 32pt
\title{{\myfont Método de eliminación nuclear}}

\maketitle

\begin{itemize}
    \item Recoger el número de células de cultivo deseado.
    \item Centrifugar en eppendorf si es posible en minifuga a 4000 rpm durante 5 min. Si no es posible, centrifugar en tubo de cristal en centrífuga de cultivos a 1500 rpm durante 5 min.
    \item Eliminar SN con mucho cuidado para no levantar el pellet.
    \item Resuspender en 100 $\mu$L de tampón A (también se puede en 50 $\mu$L de cara a no tener tanto volumen para WB. Incubar durante 15 min en hielo.
    \\
    Tampón A:
    \begin{itemize}
        \item 10 mM HEPES, pH = 7.9.
        \item 10 mM KCl.
        \item 0.1 mM EDTA.
        \item 0.1 mM EGTA.
        \item 0.5 mM PMSF.
    \end{itemize}
    \item Añadir 6.25 $\mu$L de NP-40 10 \% y mezclar bien pero con cuidado para no romper los núcleos. En el caso de haber utilizado 50 $\mu$L de tampón A, 6 $\mu$L de NP-40 10 \% es suficiente.
    \item Inmediatamente centrifugar a 13.000 rpm durante 5 min para pelletear los núcleos.
    \item Recoger el SN, que se corresponde al WCL sin núcleos.
\end{itemize}

\end{document}
